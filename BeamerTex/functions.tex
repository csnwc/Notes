\section{E: Functions}

\begin{frame}[fragile]
\frametitle{Simple Functions}
\begin{columns}

\begin{column}{0.40\textwidth}
\lstinputlisting[style=basicc]{../Code/ChapE/min2func.c}
\outputlisting{../Code/ChapE/min2func.manout}
\end{column}

\begin{column}{0.50\textwidth}
\begin{itemize}[<+->]
\item Execution begins, as normal, in the \verb^main()^ function.
\item The function {\it prototype is shown} at the top of the file.
This allows the compiler to check the code more thoroughly.
\item The function is defined between two braces.
\item The function \verb^min()^ returns an \verb^int^ and takes
two \verb^int^'s as arguments. These are copies of {\tt j} and {\tt k}.
\item The \verb^return^ statement is used to return a value
to the calling statement.
\end{itemize}
\end{column}

\end{columns}
\end{frame}

%----------------------------------------------------


\begin{frame}[fragile]
\frametitle{Call-by-Value}

\begin{columns}

\begin{column}{0.45\textwidth}
In the following example, a function is passed an integer
using call by value:
\lstinputlisting[style=basicc]{../Code/ChapE/callbyval.c}
\outputlisting{../Code/ChapE/callbyval.autoout}
\end{column}

\begin{column}{0.45\textwidth}
\begin{itemize}[<+->]
\item The function does not change the value of \verb^x^ in
\verb^main()^, since {\tt a} in the function is effectively
only a {\bf copy} of the variable.
\item A function which has no return value, is declared \verb^void^
and, in other languages, might be termed a {\it procedure}.
\item Most parameters used as arguments to functions in C are copied - this is known as {\it call-by-value}. We'll see the alternative, {\it call-by-reference}, later.
\end{itemize}
\end{column}

\end{columns}
\end{frame}

%----------------------------------------------------


% Testing
\begin{frame}[fragile]
\frametitle{Testing}
\begin{columns}

\begin{column}{0.40\textwidth}
\lstinputlisting[style=basicc]{../Code/ChapE/numfacts.c}
\end{column}

\begin{column}{0.50\textwidth}
\begin{itemize}[<+->]
\item This is a (not very good) function to compute the number of factors a number has.
\item A factor is a number by which a larger (whole/integer) number can be divided.
\item $36$ has $6$ factors: $1, 2, 3, 4, 6, 12$ and $36$ itself.
\item How do we know the program works though ?
\item Running it ? \VerbatimInput{../Code/ChapE/numfacts.autoout}
\item We need something more automated.
\end{itemize}
\end{column}

\end{columns}
\end{frame}

%----------------------------------------------------


\begin{frame}[fragile]
\frametitle{Pre- and Post-Conditions}
\begin{columns}

\begin{column}{0.40\textwidth}
\lstinputlisting[style=basicc]{../Code/ChapE/numfacts_cond.c}
\end{column}

\begin{column}{0.55\textwidth}
\begin{itemize}[<+->]
\item Pre-conditions check the inputs to functions, typically their arguments.
\item Post-conditions check the returns from functions.
\item An {\tt assert} simple states some test that {\bf ought} to be true. If not, the
program aborts with an error.
\item There's a sense that this is somehow {\it safer}, but we haven't exactly done much testing on it to ensure the correct answers are returned.
\end{itemize}
\end{column}

\end{columns}
\end{frame}

%----------------------------------------------------


\begin{frame}[fragile]
\frametitle{Assert Testing}
\begin{columns}

\begin{column}{0.40\textwidth}
\lstinputlisting[style=basicc]{../Code/ChapE/numfacts_assert.c}
\end{column}

\begin{column}{0.55\textwidth}
\begin{itemize}[<+->]
\item We will use assert testing in this style {\bf every} time we write a function.
\item These tests tend to get quite long, so we generally collect them in a function called {\tt test()} which itself is called from {\tt main()}.
\item If there is no error, there is no output from this program.
\item By \verb^#define^'ing \verb^NDEBUG^ before the {\tt \#include <assert.h>}, all assertions are ignored, allowing them to be used during development and switched off later.
\end{itemize}
\end{column}

\end{columns}
\end{frame}

%----------------------------------------------------

\begin{frame}[fragile]
\frametitle{Self-test : Multiply}
\begin{columns}

\begin{column}{0.5\textwidth}
\begin{itemize}[<+->]
\item Write a simple function \verb^int mul(int a, int b)^ which
multiples two integers together {\bf without} the use of the
multiply symbol in C (i.e. the \verb^*^)
\item Use iteration (a loop) to achieve this.
\item $7\times8$ is computed by adding up $7$ eight times.
\item Use \verb^assert()^ calls to test it thoroughly - I've given you some to get you started.
\end{itemize}
\end{column}

\begin{column}{0.40\textwidth}
\lstinputlisting[style=basicc]{../Code/ChapE/mult.c}
\end{column}

\end{columns}
\end{frame}

%----------------------------------------------------

\begin{frame}[fragile]
\frametitle{Program Layout}
\begin{columns}

\begin{column}{0.5\textwidth}
It is normal for the \verb^main()^ function to come
first in a program :
\begin{lstlisting}[style=basicc,numbers=none]
#include <stdio.h>
#include <stdlib.h>

list of function prototypes

int main(void)
{
   . . . . .
}

int f1(int a, int b)
{
   . . . . .
}

int f2(int a, int b)
{
   . . . . .
}
\end{lstlisting}
\end{column}

\begin{column}{0.5\textwidth}
However, it is theoretically possible to avoid the need for function prototypes
by defining a function before it is used :

\begin{lstlisting}[style=basicc,numbers=none]
#include <stdio.h>
#include <stdlib.h>

int f1(int a, int b)
{
   . . . . .
}

int f2(int a, int b)
{
   . . . . .
}

int main(void)
{

   . . . . .

}
\end{lstlisting}
We will {\bf never} use this second approach - put {\tt main()} first with the prototypes above it.
\end{column}

\end{columns}
\end{frame}

%----------------------------------------------------

\begin{frame}[fragile]
\frametitle{Replacing Functions with Macros}
\begin{columns}

\begin{column}{0.40\textwidth}
\lstinputlisting[style=basicc]{../Code/ChapE/min2func_def.c}
\outputlisting{../Code/ChapE/min2func_def.manout}
\end{column}

\begin{column}{0.50\textwidth}
\begin{itemize}[<+->]
\item There's sometimes a (tiny) time penalty for using functions.
\item The contents of the functions are saved onto a special stack, so that when you return to the function, its variables and state can be restored.
\item \url{https://en.wikipedia.org/wiki/Call_stack}
\item Historically, for small functions that needed to be fast, programmers might have \verb^#define^ a macro.
\item There's a problem though - what happens if we used {\tt m = MIN(i++, j++);} ?
\item This is expanded to {\tt ((i++)<(j++)?(i++):(j++))} which is {\bf not} what was intended.
\end{itemize}
\end{column}

\end{columns}
\end{frame}

%----------------------------------------------------

\begin{frame}[fragile]
\frametitle{The {\tt inline} modifier}
\begin{columns}

\begin{column}{0.50\textwidth}
\begin{itemize}[<+->]
\item In C99 the {\tt inline} modifier was introduced
\url{https://en.wikipedia.org/wiki/Inline_function}
\end{itemize}
\begin{quote}
... serves as a compiler directive that suggests (but does not
require) that the compiler substitute the body of the function inline by
performing inline expansion, i.e. by inserting the function code at the
address of each function call, thereby saving the overhead of a function
call.
\end{quote}
\end{column}

\begin{column}{0.40\textwidth}
\lstinputlisting[style=basicc]{../Code/ChapE/min2func_inline.c}
\outputlisting{../Code/ChapE/min2func_inline.manout}
\end{column}

\end{columns}
\end{frame}

%----------------------------------------------------

\begin{frame}[fragile]
\frametitle{Factorials via Iteration}
\begin{columns}

\begin{column}{0.55\textwidth}
\begin{itemize}[<+->]
\item A repeated computation computation is normally
achieved via {\it iteration}, e.g. using \verb^for()^:
\item Here we compute the factorial of a number - the factorial of
$4$, written as $4!$, is simply $4\times3\times2\times1$.
\item Obviously, we'd do more assert tests in the full verson.
\end{itemize}
\end{column}

\begin{column}{0.40\textwidth}
\lstinputlisting[style=basicc]{../Code/ChapE/iterfact.c}
\end{column}

\end{columns}
\end{frame}

%----------------------------------------------------

\begin{frame}[fragile]
\frametitle{Factorials via Recursion (Advanced)}
\begin{columns}

\begin{column}{0.60\textwidth}
\begin{itemize}[<+->]
\item We could achieve the same result using recursion.
\item The factorial of $4$ can be thought of as $4\times3!$
\item A recursive function calls {\it itself} - there may be many versions of the same function `alive' at the same time during execution.
\end{itemize}
\end{column}

\begin{column}{0.35\textwidth}
\lstinputlisting[style=basicc]{../Code/ChapE/recurfact.c}
\end{column}

\end{columns}
\end{frame}

%----------------------------------------------------

\begin{frame}[fragile]
\frametitle{Self-test : Multiply (Advanced)}
\begin{columns}

\begin{column}{0.5\textwidth}
\begin{itemize}[<+->]
\item Write a simple function \verb^int mul(int a, int b)^ which
multiples two integers together {\bf without} the use of the
multiply symbol in C (i.e. the \verb^*^)
\item Use recursion to achieve this.
\item Use \verb^assert()^ calls to test it thoroughly.
\end{itemize}
\end{column}

\begin{column}{0.35\textwidth}
\lstinputlisting[style=basicc]{../Code/ChapE/mult.c}
\end{column}

\end{columns}
\end{frame}

